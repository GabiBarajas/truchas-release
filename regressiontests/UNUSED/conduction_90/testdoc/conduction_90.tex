\documentclass[12pt]{article}

\usepackage{times}
\usepackage{latexmake}

\usepackage{amsmath}
\usepackage{amssymb}
\usepackage{amsfonts}


\textheight 9.35in \textwidth 6.8in %
\oddsidemargin -0.15in \evensidemargin -0.15in
\topmargin -15mm
\footskip 8mm
\headheight 13pt


\begin{document}

\begin{center}
{\large {\bf Discontinuous Conductivity Orthogonal to the Heat Flux}}
\end{center}

\subsubsection*{Objective}


This test problem tests the case of a discontinuous conductivity in
the heat transfer model. All parameters are chosen such that we can
test the results against an analytic solution.

\subsubsection*{Definition}

We consider the 2D conductive heat transfer problem that is defined as
follows.
$$
-\nabla\cdot K\nabla T = 0.
$$
The problem domain is the cube $[0,1]^2$, which is
discretized using an orthogonal mesh of size $10 \times
10 \times 10$. The conductivity is piecewise constant
$$
K(T) = \left\{\begin{array}{l}
              1,\quad x<0.5 \\
              1000,\quad x>0.5
             \end{array}
             \right.
$$
The boundary conditions are a constant Dirichlet boundary conditions
on the boundaries $x=0$ and $x=1$ 
\begin{eqnarray*}
T(x,y,z) &=& 0,\quad x=0 \\
T(x,y,z) &=& 100,\quad x=1 
\end{eqnarray*} 
All other boundaries have homogeneous Neumann boundary conditions
(insulation). As a result, this problem has a one-dimensional solution 
$$
T(x) = \left\{
\begin{array}{l}
\frac{1000}{5005}x,\quad x<\frac{1}{2} \\
\frac{1000000}{5005} x + \frac{499500}{5005},\quad x>\frac{1}{2}
\end{array}
\right.
$$


\subsubsection*{Metrics}

We compare the temperature at the final time step to the
analytic solution. We use the $\ell^\infty$-norm for this comparison. 

\subsubsection*{Truchas Model}

To run this example in truchas, we select the heat conduction model
exclusively and a constant time step of $0.1$.
The boundary conditions are selected as described above.
The interface between the two materials with different conductivities
is modeled using the least squares model.
The heat transfer problem is solved using Newton Krylov with flexible
GMRES as a preconditioner, which is itself preconditioned by SSOR, and
we run truchas for ten time steps. We use the ortho operator to
discretize the gradient. 


\subsubsection*{Results}

The results of these runs are compared to the analytic solution.

\end{document}